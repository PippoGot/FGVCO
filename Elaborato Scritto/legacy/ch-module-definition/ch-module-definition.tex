\chapter{Interface Definition}

%-----------------------------------------------------------------------------------------

\section{Interface Definition: Input}

\subsection*{Frequency modulation}

First thing first, we need a way to control the frequency of the VCO.
This can of course be achieved through simple knobs or sliders, however a more
interesting feature is an LFO or CV control input.

We will then take an hybrid approach by having:

\begin{itemize}
    \item Frequency modulation knobs, coarse and fine to adjust and tune the frequency
          as we please;

    \item Frequency LFO/CV gate input and attenuator knob to have a for dinamic control;
\end{itemize}

As the human ear perceives music and sound in a logarithmic fashion, we will also need
a linear to exponential converter, since an exponential voltage would be hard to manually
control.

\subsection*{Mode selection}

A VCO can be used either as a sound generator (VCO) or as a modulator (LFO), simply by
changing the range of frequencies it can outputs, therefore we can add a switch to choose
the device's functioning mode.

%-----------------------------------------------------------------------------------------

\section{Interface Definition: Output}

\subsection*{Waveforms}

As we've already discussed, a VCO could theoretically give us almost any waveform,
if designed to do so, obviously. However we will limit the outputs to the five fundamentals
wave shapes:

\begin{itemize}
    \item Sine wave;
    \item Triangle wave;
    \item Square wave;
    \item Ramp wave;
    \item Sawtooth wave;
\end{itemize}

With those five basic waveforms we can then get really far with all the available effects
and modulations that other modules can offer.

We would like parallel aoutputs (although you can certainly get an analog multiplexer
to have only one output channel instead of five), and that all of our waves have the same
frequency.

Also we would like all of our outputs to have an attenuator just to give us the possibility
to adjust the amplitude of the signal before going inside another module.

\subsection*{Other miscellaneous outputs}

A sync pulse may come in handy to trigger effects or with a PLL module. We can also add
an LED to see the state of the pulse.

LEDs showing the current function mode can also come in handy to quickly see how the
module is behaving.

%-----------------------------------------------------------------------------------------

\section{Module Block Diagram}

Summing all of the requirements described in the previous chapters we can now draw a
preliminary block diagram to better understand how we need to tie everything together.

\begin{figure}[ht]
    \centering
    \def\svgwidth{\columnwidth}
    \input{img/full-block-diagram.pdf_tex}

    \caption{Full Module Block Diagram}
    \label{full-diagram}
\end{figure}