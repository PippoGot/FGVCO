\chapter*{Before you read!}

%-----------------------------------------------------------------------------------------

\section*{What is this manual about?}

The purpose of this manual is to explain in every aspect the design of different diy
eurorack modules, more specifically, every design choice and how every circuit works.
Ultimately the reader will be able to understand how the circuits are built, how they work
and how they can be replicated and/or modified as needed.

The reader is assumed to already have a basic understanding of electronics and electronic
components.

Finally, almost every module could be easily built with digital microcontrollers or DSPs,
although the challenge here is to take as much as possible an analog approach.

%-----------------------------------------------------------------------------------------

\section*{Notation and Terminology}

In order to better comprehend what is going on, there are going to be a lot of drawings,
schematics and diagrams in this manual, so here are some tips on how to interpret what you
see:

\begin{itemize}
      \item
            \textbf{\textcolor{magenta}{Audio signal}}: a signal that can be fed into an
            amplifier. Ultimately almost every synth module is built to create and modify
            this kind of signals (with a few exceptions);
      \item
            \textbf{\textcolor{orange}{Digital signal}}: a signal is either logically high
            (higher than a fixed value), or logically low (lower than a fixed value).
            These kind of signals are used to control various different modules functions,
            such as gates for envelope generators, clocks for step sequencers and so on;
      \item
            \textbf{\textcolor{cyan}{Control voltage}}: an analog signal that can assume
            every value between two different fixed references, often $\pm V_{cc}$. Much
            like a digital signal, a control voltage is used to control modules' functions,
            but it can controls a specific parameter with a much higher flexibility and with
            particular relationships (such as linear or exponential).
\end{itemize}

Usually the colors will be used in block diagram's arrows to show what kind of signal they
bring, like this:

\begin{tikzpicture}
      \synthmodule{This is the Input}{MODULE}{This is the Output};

      \draw[cv-arrow] (input) -- (module);
      \draw[audio-arrow] (module) -- (output);
\end{tikzpicture}

As for the terminology:

\begin{itemize}
      \item
            \textbf{Interface}: it refers to the front panel components such as knobs,
            connectors, LEDs and everything that can be manipulated from the final user.
      \item
            \textbf{Core}: it refers to all the circuitry that makes the magic happens. It
            is usually kept behind the interface board.
\end{itemize}

%-----------------------------------------------------------------------------------------

\section*{Structure}

Every module will be treated as follows:

\begin{enumerate}
      \item
            General introduction and presentation of the module (I/O configuration,
            function principle and possible variants);
      \item
            Project specifics and general block diagram of the module;
      \item
            General blocks analysis and implementation;
      \item
            Final choice of the circuit, final complete block diagram and front panel design;
\end{enumerate}
