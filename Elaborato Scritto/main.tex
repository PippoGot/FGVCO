\documentclass[twoside, 12pt]{report}

% --- PACCHETTI ---

% unità di misura e matematica
\usepackage{siunitx, amsmath}

% titolo capitoli e sezioni personalizzato
\usepackage{titlesec}

\titleformat{\chapter}
{\normalfont\LARGE\bfseries}{\thechapter.}{10pt}{\LARGE}
\titlespacing*{\chapter}{0pt}{0pt}{20pt}

\titleformat{\section}
{\normalfont\large\bfseries}{\thesection}{10pt}{\large}
\titlespacing*{\section}{0pt}{0pt}{10pt}

% include pdfs
\usepackage{pdfpages}

% grafiche e tabelle
\usepackage{graphicx, calc}

\graphicspath{{images/}}

\usepackage{tikz, pgfplots, pgfplotstable}

\pgfplotsset{compat = newest}
\usepgfplotslibrary{units}

\usepackage{array, csvsimple}

\newcolumntype{C}{>{\centering}m{1.8cm}}
\newcolumntype{L}{>{\centering\arraybackslash}m{1.8cm}}

\usepackage{float}

% colori personalizzati
\usepackage{colortbl, xcolor}

% normali
\definecolor{myBlue}{RGB}{1, 187, 235}
\definecolor{myRed}{RGB}{255, 43, 43}
\definecolor{myYellow}{RGB}{255, 221, 86}
\definecolor{myBlack}{RGB}{28, 28, 28}
\definecolor{myGrey}{RGB}{134, 134, 134}
% chiari
\definecolor{myLightBlue}{RGB}{177, 239, 255}
\definecolor{myLightRed}{RGB}{255, 177, 177}
\definecolor{myLightYellow}{RGB}{255, 240, 180}
\definecolor{myLightGrey}{RGB}{218, 218, 218}

\pgfplotscreateplotcyclelist{modular}{
        {myBlue},
        {myRed},
        {myBlack},
        {myGray},
        {myYellow},
      }

\usepackage{subcaption, caption}

% numerazione capitoli in italiano
\usepackage[italian]{babel}

% spaziatura
\usepackage{parskip}

% impaginazione
\usepackage{setspace} % For doublespacing

\doublespacing

\usepackage{geometry}

\geometry{a4paper, top = 2cm, bottom = 2cm, outer = 2cm, inner = 3cm, heightrounded}

% font
\usepackage{times}

% bibliografia
\usepackage{csquotes, biblatex}
\addbibresource{bibliography.bib}

\usepackage{xurl, url} % evita che la bibliografia esca dal margine

\urlstyle{same}
\appto{\bibsetup}{\raggedright}

% salta pagine bianche
\usepackage{afterpage}

\newcommand\emptypage{
    \newpage
    \thispagestyle{empty}
    \mbox{}
    \newpage
}

% interlinea
\linespread{1.5}

% comandi per aggiungere chapter, section e subsection non numerate al toc
\newcommand\unchapter[1]{
    \chapter*{#1}
    \addcontentsline{toc}{chapter}{#1}}

\newcommand\unsection[1]{
    \section*{#1}
    \addcontentsline{toc}{section}{#1}}

\newcommand\unsubsection[1]{
        \subsection*{#1}
        \addcontentsline{toc}{subsection}{#1}}

% --- DOCUMENTO PRINCIPALE ---

% inizio del documento
\begin{document}

% --- FRONTESPIZIO ---
% DA MODIFICARE DATA DI LAUREA !!!
\includepdf{pdfs/frontespizio/frontespizio_tesi.pdf}
\emptypage

% --- INDICE ---

\tableofcontents
\emptypage

% --- CAPITOLI ---

\unchapter{Introduzione e Specifiche di Progetto}

%--------------------------------------------------------------------------------------------

In questa tesi viene discussa la progettazione di un sintetizzatore musicale compatibile con
lo standard modulare più diffuso al giorno d'oggi, ovvero Eurorack \cite{eurorack}. Più
precisamente si vuole realizzare un generatore di segnali che offra la possibilità di essere
controllato in tensione (Voltage Controlled Oscillator, o VCO in breve). In questo modo,
applicando dei segnali variabili nel tempo in ingresso al modulo, si è in grado di variare
dinamicamente la frequenza dei segnali in uscita.

Il range scelto per tale variazione comprende la fascia di frequenze dello spettro audio,
quindi da poche decine di $Hz$ a circa $7kHz$. Si desidera inoltre, la possibilità di
convertire a piacere il funzionamento del modulo in oscillatore a bassa frequenza (Low Frequency
Oscillator, o LFO), spostando quindi il range di frequenze disponibili da frazioni di
$Hz$ a qualche decina di $Hz$.
\medskip

\begin{figure}[ht]
    \centering
    \includegraphics{graphs/functioning_range.png}
    \caption{Range di funzionamento approssimato, in scala logaritmica}
    \label{functioning_range}
\end{figure}

In modalità VCO quindi, il modulo produrrà dei segnali che attraverso un adeguato sistema
potranno essere ascoltati, mentre in modalità LFO il circuito produrrà dei segnali lentamente
variabili nel tempo, utili per la modulazione e il controllo di diversi parametri in altri
moduli eventualmente presenti nel sistema.

Le forme d'onda desiderate sono quelle base, ovvero:

\begin{itemize}
    \item Sinusoide;
    \item Onda Quadra;
    \item Triangolo;
    \item Rampa;
    \item Dente di sega (sebbene nel range VCO non risulti particolarmente differente dalla
          rampa in termini di suono, per quanto riguarda il funzionamento LFO la differenza
          è radicale, poichè il segnale viene solitamente utilizzato come modulante);
\end{itemize}

inoltre, come verrà illustrato più avanti, risulta piuttosto semplice anche estrarre un
segnale a impulso, anch'esso alla stessa frequenza di quelli già generati, per il controllo
di altri moduli o parametri.
\smallskip

Per quanto riguarda le specifiche sui livelli di tensione, si vogliono imporre i seguenti
intervalli di valori:

\begin{itemize}
    \item Segnali audio (i 5 elencati poco sopra): $\pm5V$;
    \item Segnali logici (impulso): $(0V,5V)$;
    \item Segnali di controllo (ingresso): $(0V,8V)$ in modalità $1V/Octave$, ovvero
          facendo in modo che ad un incremento di $1V$ corrisponda un raddoppio di frequenza,
          cioè un'ottava;
    \item Alimentazioni: $\pm12V$ e $+5V$;
\end{itemize}

Le specifiche sopra riportate sono prese dallo standard Eurorack.

Si desidera inoltre aggiungere delle manopole per il controllo del "volume" dei segnali
in ingresso e uscita, (ad eccezione dell'impulso) e delle manopole per il controllo
manuale della frequenza.

Mettendo assieme tutti questi dettagli possiamo abbozzare una interfaccia utente,
riportata in figura \ref{panel_explained}, in modo da rendere più chiaro al lettore il
prodotto finale.
\medskip

\begin{figure}[ht]
    \centering
    \includegraphics{misc/panel_explained.png}
    \caption{Pannello frontale del modulo}
    \label{panel_explained}
\end{figure}

Si decide di realizzare l'intero circuito senza l'utilizzo di microcontrollori o sistemi
programmabili. Tale scelta viene presa per mettere alla prova più competenze possibili tra
quelle acquisite durante gli anni di studio. Una soluzione possibile sarebbe infatti
impiegare un microcontrollore con dei campioni digitali molto fitti delle forme d'onda
desiderate.

%--------------------------------------------------------------------------------------------

\chapter{Generazione dei Segnali Principali}\label{segnali_principali}

%--------------------------------------------------------------------------------------------

Si comincia dalla generazione dei segnali a rampa e a triangolo, per i quali si decide di
procedere per via digitale, utilizzando dei contatori binari abbinati ad un convertitore
digitale-analogico.

%--------------------------------------------------------------------------------------------

\section{Rampa}

%--------------------------------------------------------------------------------------------

\subsection*{Principio di Funzionamento}

%--------------------------------------------------------------------------------------------

Per il segnale a rampa si fa uso di un contatore unidirezionale, ovvero un dispositivo
in grado di contare automaticamente da $0$ a $2^n$ (con $n$ numero di bit) semplicemente
fornendo un segnale di clock adeguatamente dimensionato. Maggiore il numero di bit,
maggiore la precisione del nostro segnale, quindi minore l'intensità del rumore generato.

\begin{figure}[H]
    \centering

    \begin{subfigure}{.5\textwidth}
        \centering
        \includegraphics{graphs/low_res_ramp.png}
        \caption{Rampa ottenuta con un contatore a 2 bit}
        \label{low_res_ramp}
    \end{subfigure}%
    \begin{subfigure}{.5\textwidth}
        \centering
        \includegraphics{graphs/high_res_ramp.png}
        \caption{Rampa ottenuta con un contatore a 8 bit}
        \label{high_res_ramp}
    \end{subfigure}

    \caption{Confronto tra contatori unidirezionali con diverso numero di bit}
    \label{ramps}
\end{figure}

Tuttavia aumentando il numero di bit del contatore è facile intuire che, a parità di
frequenza del segnale in uscita, la frequenza del segnale di clock debba necessariamente
aumentare, vale infatti la seguente relazione:

\begin{equation}\label{fsignal}
    f_{signal}=\frac{f_{clk}}{2^n}\ [Hz]
\end{equation}

poichè il contatore deve effettuare un conteggio completo durante un periodo del segnale
in uscita. Questo implica dunque un limite massimo al numero di bit del contatore.

Il numero di bit utilizzati per la nostra applicazione è pari a $8$, valore che ci consente
di limitare al $MHz$ la frequenza di clock, contare fino a $255$ e dividere l'intervallo di
tensione d'uscita in altrettanti livelli, ottenendo quindi una variazione di

\begin{equation}\label{vstep}
    V_{step}=\frac{2\cdot V_{ref}}{2^n}=\frac{10}{256}\approx39\ mV
\end{equation}

per ogni singolo bit (scegliendo $V_{ref}=+5\ V$).

\begin{figure}[H]
    \centering
    \includegraphics{block_diagrams/ramp_block_diagram.png}
    \caption{Schema a blocchi del sottosistema per la generazione della rampa}
    \label{ramp_block_diagram}
\end{figure}

A questo punto possiamo calcolare le specifiche del segnale di clock da generare,
andando a vedere quali sono le frequenze desiderate per i segnali audio:

\begin{itemize}
    \item Valore minimo (nota A0): $f_{signal-min}=27.5\ Hz\rightarrow f_{clk-min}\approx7\ kHz$
          a cui corrisponderà un ingresso di $0\ V$;
    \item Valore massimo (nota A8): $f_{signal-max}\approx7\ kHz\rightarrow f_{clk-max}\approx1.8\ MHz$
          a cui corrisponderà un ingresso di $+8\ V$;
\end{itemize}

quindi un range di funzionamento esteso lungo 8 ottave.

%--------------------------------------------------------------------------------------------

\subsection*{Componenti Utilizzati e Schemi Elettrici}

%--------------------------------------------------------------------------------------------

I componenti utilizzati per la realizzazione del circuito sono un 74HC590 \cite{74hc590},
contatore binario a 8 bit, e un DAC0800 \cite{dac0800}, convertitore digitale-analogico, sempre
a 8 bit.

Per il circuito DAC si utilizza lo schema a pg.10 del relativo datasheet del componente.
Tale configurazione ci permette di convertire il dato binario in un valore compreso
nell'intervallo $\pm V_{ref}$, tuttavia si utilizzano un amplificatore operazionale e dei
resistori di valore differente (rispettivamente TL074 \cite{tl074} e $R_L=\bar{R_L}=3.3\ k\Omega$)
per questioni di disponibilità. Si noti che anche $V_{ref}$ viene scelta diversa rispetto
allo schema nel datasheet (ovvero $+5\ V$), in modo da garantire le specifiche di progetto
sul segnale in uscita.

\begin{figure}[H]
    \centering
    \includegraphics{circuits/DAC_circuit.png}
    \caption{Schema elettrico del DAC ($\pm V_{cc}=\pm 12\ V$)}
    \label{DAC_circuit}
\end{figure}

Il DAC eroga una corrente $I_{out}$ proporzionale all'ingresso digitale $x$, che viene poi
convertita in una tensione con un operazionale. Le due grandezze sono legate dalla relazione:

\begin{equation}\label{vout_dac}
    V_{out}=V_{ref}\left(\frac{2x-255}{256}\right)=5\left(\frac{2x-255}{256}\right)\ [V]
\end{equation}

Il contatore invece viene collegato nel seguente modo:

\begin{figure}[H]
    \centering
    \includegraphics{circuits/ramp_counter_circuit.png}
    \caption{Schema elettrico del contatore per l'onda a rampa ($+V_{dd}=+5\ V$)}
    \label{ramp_counter_circuit}
\end{figure}

Si noti l'uscita $RCO$ in figura \ref{ramp_counter_circuit} dalla quale viene prelevato
il segnale per la generazione dell'impulso, che verrà discusso più in dettaglio nel capitolo
\ref{segnali_secondari}.

Infine, collegando i due blocchi insieme, l'andamento di $V_{out}$ sarà simile a quello
rappresentato in figura \ref{high_res_ramp}, e ad ogni impulso di clock corrisponderà un
gradino di tensione di circa $40\ mV$ come calcolato con la formula \ref{vstep}.

%--------------------------------------------------------------------------------------------

\subsection*{Risultati Pratici}

%--------------------------------------------------------------------------------------------

Verifichiamo quindi la correttezza del circuito realizzato con il setup di misura in figura
\ref{mis_ramp}.

\begin{figure}[H]
    \centering

    \begin{subfigure}{.5\textwidth}
        \centering
        \includegraphics{block_diagrams/mis_ramp.png}
        \caption{Setup di misura}
        \label{mis_ramp}
    \end{subfigure}%
    \begin{subfigure}{.5\textwidth}
        \centering
        \includegraphics[scale = 0.2]{acquisitions/ramp_wave.png}
        \caption{Acquisizione del segnale ottenuto}
        \label{acq_ramp}
    \end{subfigure}

    \caption{Funzionamento del circuito per il segnale a rampa}
    \label{acq_ramp_signals}
\end{figure}

Si apprezza come l'onda generata abbia esattamente la forma voluta, descritta nel
principio di funzionamento.

%--------------------------------------------------------------------------------------------

\section{Triangolo}

%--------------------------------------------------------------------------------------------

\subsection*{Principio di Funzionamento}

%--------------------------------------------------------------------------------------------

Il ragionamento è del tutto analogo a quello del contatore per la rampa, tuttavia in questo
caso il contatore utilizzato è bidirezionale (quindi in grado di contare da $0$ a $2^n$ e
viceversa) e necessita di un segnale che determini la direzione di conteggio (Up o Down).

\begin{figure}[H]
    \centering
    \includegraphics{block_diagrams/triangle_block_diagram.png}
    \caption{Schema a blocchi del sottosistema per la generazione del triangolo}
    \label{triangle_block_diagram}
\end{figure}

\begin{figure}[H]
    \centering

    \begin{subfigure}{.5\textwidth}
        \centering
        \includegraphics{graphs/low_res_triangle.png}
        \caption{Triangolo ottenuto con un contatore a 2 bit}
        \label{low_res_triangle}
    \end{subfigure}%
    \begin{subfigure}{.5\textwidth}
        \centering
        \includegraphics{graphs/high_res_triangle.png}
        \caption{Triangolo ottenuto con un contatore a 8 bit}
        \label{high_res_triangle}
    \end{subfigure}

    \caption{Confronto tra contatori bidirezionali con diverso numero di bit}
    \label{triangles}
\end{figure}

La configurazione del DAC rimane quella utilizzata per la rampa e rappresentata in figura
\ref{DAC_circuit}. In questo caso tuttavia il numero di cicli di clock utilizzati è doppio
rispetto a quello per la rampa, poichè dovranno essere eseguiti 256 conteggi verso l'alto
e altri 256 verso il basso per ottenere un singolo periodo di onda triangolare. Ne
consegue quindi che anche la frequenza di clock in ingresso a questo sottosistema dovrà
essere doppia rispetto alla rampa, come risulta evidente in figura \ref{steps}.

\begin{figure}[H]
    \centering

    \begin{subfigure}{.5\textwidth}
        \centering
        \includegraphics{graphs/low_res_ramp.png}
        \caption{Rampa ottenuta con un contatore a 2 bit}
    \end{subfigure}%
    \begin{subfigure}{.5\textwidth}
        \centering
        \includegraphics{graphs/low_res_triangle.png}
        \caption{Triangolo ottenuto con un contatore a 2 bit}
    \end{subfigure}

    \caption{Confronto del conteggio tra contatori unidirezionali (a) e bidirezionali (b)}
    \label{steps}
\end{figure}

%--------------------------------------------------------------------------------------------

\subsection*{Componenti Utilizzati e Schemi Elettrici}

%--------------------------------------------------------------------------------------------

L'unico componente diverso rispetto al circuito per la rampa è il contatore, che come già
detto deve essere bidirezionale. Si utilizzano due 74LS169 \cite{74ls169} in cascata nella
seguente configurazione:

\begin{figure}[H]
    \centering
    \includegraphics{circuits/triangle_counter_circuit.png}
    \caption{Schema elettrico dei contatori per l'onda triangolare ($+V_{dd}=+5\ V$)}
    \label{triangle_counter_circuit}
\end{figure}

Il componente utilizzato presenta anche degli ingressi per il preset del numero di partenza
(pin da 3 a 6), che però nel nostro caso non vengono utilizzati.

L'uscita denominata $RCO2$ verrà utilizzata per pilotare il verso del conteggio, essa infatti
commuta durante il ciclo di clock corrispondente al conteggio precedente all'overflow, ovvero
$255$ in modalità "Up" e $0$ in modalità "Down".

\begin{figure}[H]
    \centering
    \includegraphics{graphs/RCO2_behaviour.png}
    \caption{Andamento teorico del segnale $RCO2$ in funzione del conteggio}
    \label{RCO2_behaviour}
\end{figure}

%--------------------------------------------------------------------------------------------

\subsection*{Risultati Pratici}

%--------------------------------------------------------------------------------------------

Si verifica ora la correttezza del circuito realizzato.

\begin{figure}[H]
    \centering

    \begin{subfigure}{.5\textwidth}
        \centering
        \includegraphics{block_diagrams/mis_triangle.png}
        \caption{Setup di misura}
        \label{mis_triangle}
    \end{subfigure}%
    \begin{subfigure}{.5\textwidth}
        \centering
        \includegraphics[scale = 0.2]{acquisitions/triangle_wave.png}
        \caption{Acquisizione dei segnali ottenuti}
        \label{acq_triangle}
    \end{subfigure}

    \caption{Funzionamento del circuito per il segnale a triangolo}
    \label{acq_triangle_signals}
\end{figure}

Anche in questo caso si apprezza che quanto discusso finora corrisponde al reale comportamento
del circuito.

%--------------------------------------------------------------------------------------------

\section{Adattamento dei Segnali di Clock e Pilotaggio}

%--------------------------------------------------------------------------------------------

Si è visto come, per avere la stessa frequenza di segnale d'uscita, il contatore del
triangolo deve avere una frequenza di clock doppia rispetto a quella del contatore della
rampa. Questo problema si risolve facilmente inserendo prima del contatore unidirezionale
un divisore di frequenza, ottenuto con un semplice toggle flip-flop (TFF).

\begin{figure}[H]
    \centering

    \begin{subfigure}{.5\textwidth}
        \centering
        \includegraphics{block_diagrams/freq_divider_block_diagram.png}
        \caption{Schema a blocchi}
        \label{freq_divider_block_diagram}
    \end{subfigure}%
    \begin{subfigure}{.5\textwidth}
        \centering
        \includegraphics{graphs/clock_divider_behaviour.png}
        \caption{Andamento dei segnali previsto}
        \label{clock_divider_behaviour}
    \end{subfigure}

    \caption{Divisore di frequenza}
\end{figure}

Le specifiche sul segnale di clock ci impongono allora di generare un segnale a onda rettangolare
con frequenza variabile nel range $(14\ kHz\div3.6\ MHz)$, in modo che ½$f_{clk}$ abbia
i valori di frequenza inizialmente calcolati.

Per fare in modo che il contatore del triangolo cambi effettivamente verso di conteggio invece,
è necessario impiegare un altro TFF utilizzando come ingresso $\overline{RCO2}$ (poichè
attivo a livello logico basso) e connesso all'ingresso $U/D$ del contatore.

\begin{figure}[H]
    \centering

    \begin{subfigure}{.5\textwidth}
        \centering
        \includegraphics{block_diagrams/UD_block_diagram.png}
        \caption{Schema a blocchi}
        \label{UD_block_diagram}
    \end{subfigure}%
    \begin{subfigure}{.5\textwidth}
        \centering
        \includegraphics{graphs/UD_behaviour.png}
        \caption{Andamento dei segnali previsto}
        \label{UD_behaviour}
    \end{subfigure}

    \caption{Pilotaggio del contatore}
\end{figure}

I componenti utilizzati per lo scopo sono un 74HC73 \cite{74hc73}, che fornisce i due
flip-flop necessari, e un 2N7000 \cite{2n7000}, mosfet a canale N con tensioni di soglia
compatibili con i livelli di tensione presenti nel circuito ($+5\ V$).

Lo schema elettrico per l'inverter è rappresentato in figura \ref{inverter_circuit}.

\begin{figure}[H]
    \centering
    \includegraphics{circuits/inverter_circuit.png}
    \caption{Schema elettrico dell'inverter logico ($+V_{dd}=+5\ V$)}
    \label{inverter_circuit}
\end{figure}

%--------------------------------------------------------------------------------------------

\subsection*{Risultati Pratici}

%--------------------------------------------------------------------------------------------

Si verifica la correttezza dei circuiti:

\begin{figure}[H]
    \centering

    \begin{subfigure}{.5\textwidth}
        \centering
        \includegraphics{block_diagrams/mis_clock_divider.png}
        \caption{Setup di misura}
        \label{mis_clock_divider}
    \end{subfigure}%
    \begin{subfigure}{.5\textwidth}
        \centering
        \includegraphics[scale = 0.2]{acquisitions/clock_wave.png}
        \caption{Acquisizione dei segnali $f_{clk}$ e ½$f_{clk}$}
        \label{acq_clock_divider}
    \end{subfigure}

    \caption{Funzionamento del circuito divisore di frequenza}
    \label{clock_divider}
\end{figure}

\begin{figure}[H]
    \centering

    \begin{subfigure}{.5\textwidth}
        \centering
        \includegraphics{block_diagrams/mis_UD.png}
        \caption{Setup di misura}
        \label{mis_UD}
    \end{subfigure}%
    \begin{subfigure}{.5\textwidth}
        \centering
        \includegraphics[scale = 0.2]{acquisitions/UD_wave.png}
        \caption{Acquisizione dei segnali $RCO2$ e $U/D$}
        \label{acq_UD}
    \end{subfigure}

    \caption{Funzionamento del circuito di pilotaggio del contatore up-down}
    \label{UD}
\end{figure}

I comportamenti coincidono con quanto descritto nelle figure \ref{clock_divider_behaviour} e
\ref{UD_behaviour}.

%--------------------------------------------------------------------------------------------

\chapter{Convertitore Tensione-Frequenza}

%--------------------------------------------------------------------------------------------

Si vuole ora progettare la sezione per la generazione del segnale di clock, con le specifiche
ottenute dal capitolo precedente, ovvero:

\begin{itemize}
    \item Frequenza minima: $\approx 14kHz$;
    \item Frequenza massima: $\approx 3.6MHz$;
    \item Livello logico basso: $0V$;
    \item Livello logico alto: $+5V$;
\end{itemize}

%--------------------------------------------------------------------------------------------

\subsection*{Principio di Funzionamento}

%--------------------------------------------------------------------------------------------

Ciò di cui abbiamo bisogno è un circuito in grado di convertire una tensione in un segnale
a onda quadra con frequenza proporzionale alla tensione stessa, ovvero un convertitore
tensione-frequenza.

In commercio è possibile trovare chip in grado di svolgere questa funzione con l'aggiunta
di una manciata di componenti di contorno, anche se la maggior parte di questi non
arriva a coprire l'intero range di funzionamento di cui abbiamo bisogno (come ad esempio
il noto LM331 \cite{lm331}). Nel nostro caso si utilizza un VFC110 \cite{vfc110},
circuito integrato che vanta un'ottima linearità e in grado di arrivare a fornire una
frequenza in uscita di $4MHz$ in corrispondenza di una tensione di ingresso di $10V$,
esattamente ciò che la nostra applicazione richiede.
\medskip

\begin{figure}[ht]
    \centering
    \includegraphics{circuits/vfc110_internal.png}
    \caption{Estratto della struttura interna di un VFC110}
    \label{vfc100_internal}
\end{figure}

Il circuito fa uso di un integratore, la quale uscita è proporzionale alla carica immagazzinata
in $C_{int}$. Una tensione in ingresso $V_{in}$ sviluppa una corrente $I_{in}=\frac{V_{in}}{R_{in}}$
e che viene forzata in $C_{int}$, caricandolo e causando l'uscita dell'integratore a diminuire
linearmente. Non appena l'uscita dell'integratore arriva a $0V$, il comparatore scatta, attivando
il timer one-shot. Quindi un generatore di corrente $I_{ref}$ (circa $1mA$) viene connesso
all'uscita dell'integratore durante il periodo del timer $T_{OS}$ causando l'uscita dell'integratore
a crescere linearmente fino alla fine di $T_{OS}$. Successivamente il ciclo ricomincia.

L'oscillazione è regolata dall'equilibrio tra corrente in ingresso $I_{in}$ e la corrente di
reset media.
\medskip

\begin{figure}[ht]
    \centering
    \includegraphics{circuits/VFC_circuit.png}
    \caption{Schema elettrico del VFC110 utilizzato}
    \label{VFC_circuit}
\end{figure}

Per uno studio più approfondito sul funzionamento del componente si rimanda al datasheet
del componente, dal quale si ricava anche la configurazione del circuito da utilizzare
per sfruttare l'intero range offerto, modificando però i valori di alimentazione con
quelli dello standard scelto ($\pm 12V$).
\medskip

% schema elettrico

Si noti che gli unici componenti aggiunti sono condensatori di filtro e un resistore di
pull-up per l'uscita a collettore aperto.

Le relazioni tra le grandezze in gioco sono le seguenti:

$$
    I_{in}=I_{ref}\cdot\delta
    \rightarrow
    \delta=\frac{I_{in}}{I_{ref}}=\frac{V_{in}}{R_{in}\cdot I_{ref}}
$$

$$
    \frac{V_{in}}{R_{in}}=I_{ref}\cdot f_{out}\cdot T_{OS}
    \rightarrow
    f_{out}=\frac{V_{in}}{R_{in}\cdot I_{ref}\cdot T_{OS}}=\frac{\delta}{T_{OS}}
$$

%--------------------------------------------------------------------------------------------

\subsection*{Risultati Pratici e Misure}

%--------------------------------------------------------------------------------------------

% grafica setup di misura

\begin{figure}[ht]
    \centering
    \includegraphics{misc/oscilloscope_placeholder.png}
    \caption{Acquisizione dell'uscita dell'integratore (pin 12) e $f_{out}$ corrispondente}
    \label{acq_vfc110}
\end{figure}

% misura di Tos

% grafico V_in/duty (misurato e teorico) e tabella valori
% grafico V_in/f_out (misurato e teorico) e tabella valori

\chapter{Condizionamento dell'Ingresso}

%--------------------------------------------------------------------------------------------

\section{Convertitore Lineare-Esponenziale}

%--------------------------------------------------------------------------------------------

Vogliamo ora analizzare il circuto che soddisfa la specifica sulla modalità $1\ V/Octave$
dell'ingresso, ovvero il circuito in grado di convertire una tensione lineare in una
esponenziale.

La soluzione utilizzata è molto diffusa in questo tipo di applicazioni, si può infatti
trovare in molti siti di DIY come ad esempio quello di René Schmitz \cite{expo_converter},
personaggio molto noto tra gli appassionati di sintetizzatori musicali fai-da-te.

%--------------------------------------------------------------------------------------------

\subsection*{Analisi del Circuito}

%--------------------------------------------------------------------------------------------

Per l'applicazione si sfrutta la caratteristica esponenziale intrinseca del transistor
bipolare, ovvero:

\begin{equation}\label{transistor_current}
    I_e\approx I_c=I_se^{\left(\frac{V_{be}}{V_T}-1\right)}
    \approx I_se^{\left(\frac{V_{be}}{V_T}\right)}\ [A]
\end{equation}

\begin{figure}[H]
    \centering
    \includegraphics{circuits/single_transistor_circuit.png}
    \caption{BJT}
    \label{bjt}
\end{figure}

dove $V_T$ (potenziale termico) e $I_s$ (corrente di saturazione) sono variabili in
funzione della temperatura. Nella nostra analisi $V_T$ verrà considerato di valore costante
pari a $26\ mV$, mentre si rimuove dall'equazione $I_s$ collegando una coppia di transistor
(idealmente nello stesso chip, in modo che siano il più possibile simili tra loro e
termicamente accoppiati) in configurazione differenziale:

\begin{figure}[H]
    \centering
    \includegraphics{circuits/differential_pair_circuit.png}
    \caption{BJT a coppia differenziale}
    \label{differential_pair_circuit}
\end{figure}

in questo modo possiamo scrivere la seguente relazione:

\begin{equation}\label{differential_pair}
    \frac{I_{c2}}{I_{c1}}=\frac{I_s e^{\left(\frac{V_{be2}}{V_T}\right)}}{I_s e^{\left(\frac{V_{be1}}{V_T}\right)}}
    \qquad
    \rightarrow
    \qquad
    I_{c2}=I_{c1}e^{\left(\frac{V_{be2}-V_{be1}}{V_T}\right)}=I_{c1}e^{\left(\frac{V_{b2}-V_{b1}}{V_T}\right)}\ [A]
\end{equation}

dove risulta evidente che la dipendenza da $I_s$ viene completamente rimossa.

A questo punto, rinominiamo le grandezze come segue:

\begin{equation}\label{renamed_differential_pair}
    \left\{ \begin{aligned}
        I_{c2} & = I_{freq} \\
        I_{c2} & = I_{ref}
    \end{aligned} \right.
    \qquad
    \rightarrow
    \qquad
    I_{freq}=I_{ref}e^{\left(\frac{V_{b2}-V_{b1}}{V_T}\right)}\ [A]
\end{equation}

e aggiungiamo al circuito

\begin{itemize}
    \item un amplificatore invertente per portare $V_{in}$ in un range appropriato per la base
          di $Q_1$ (operazionale di sinistra, figura \ref{exponential_converter_circuit})
          \begin{equation}\label{amplifier}
              V_{b1}=-V_{in}\cdot s=
              -V_{in}\cdot\frac{R_f}{R_{in}}\cdot\frac{\%R_{pot}+R}{R_{pot}+R}\ [V]
          \end{equation}
    \item un anello di controllo per mantenere la corrente di riferimento $I_{ref}$ costante
          (operazionale centrale, figura \ref{exponential_converter_circuit})
          \begin{equation}\label{iref}
              I_{ref}=\frac{V_{HR}-V_{LR}}{R_{ref}}\ [A]
          \end{equation}
    \item un convertitore corrente-tensione al collettore di $Q_2$, per avere un segnale in
          tensione come uscita (operazionale di destra, figura \ref{exponential_converter_circuit})
          \begin{equation}\label{ivconv}
              V_{exp}=I_{freq}\cdot R_{conv}\ [V]
          \end{equation}
\end{itemize}

ottenendo quindi il seguente circuito con la relativa relazione ingresso/uscita:

\begin{figure}[H]
    \centering
    \includegraphics{circuits/exponential_converter_circuit.png}
    \caption{Schema elettrico del convertitore tensione lineare-esponenziale}
    \label{exponential_converter_circuit}
\end{figure}

\begin{equation}\label{expo_converter}
    V_{exp}=R_{conv}\cdot \frac{V_{HR}-V_{LR}}{R_{ref}}e^{\left(\frac{s\cdot V_{in}}{V_T}\right)}\ [V]
\end{equation}

%--------------------------------------------------------------------------------------------

\subsection*{Dimensionamento e Scelta dei Componenti}

%--------------------------------------------------------------------------------------------

Passiamo quindi al dimensionamento dei componenti, in modo da imporre al circuito il
comportamento voluto.

Come prima cosa calcoliamo il valore del guadagno $s$ dell'amplificatore invertente.
Si vuole:

\begin{equation}\label{iref_doubling}
    I_{freq}=I_{ref}e^{\left(\frac{s\cdot V_{in}}{V_T}\right)}
    \qquad
    \xrightarrow{+\Delta V_{in}}
    \qquad
    2I_{freq}=I_{ref}e^{\left(\frac{s\cdot(V_{in}+\Delta V_{in})}{V_T}\right)}
\end{equation}

qundi un raddoppio della corrente $I_{freq}$ per ogni variazione $\Delta V_{in}=1\ V$.
Allora possiamo riscrivere la relazione nel seguente modo:

\begin{equation}\label{s1}
    \frac{2I_{freq}}{I_{freq}}=
    \frac{I_{ref}e^{\left(\frac{s\cdot(V_{in}+\Delta V_{in})}{V_T}\right)}}
    {I_{ref}e^{\left(\frac{s\cdot V_{in}}{V_T}\right)}}
    \
    \rightarrow
    \
    2=e^{\left(\frac{s\cdot\Delta V_{in}}{V_T}\right)}
    \
    \rightarrow
    \
    ln(2)=\frac{s\cdot\Delta V_{in}}{V_T}
    \
    \rightarrow
    \
    s=\frac{V_T\cdot ln(2)}{\Delta V_{in}}
\end{equation}

\begin{equation}\label{s2}
    s=\frac{26\ mV\cdot 0.6931}{1\ V}\approx0.018\approx\frac{1}{55.5}
\end{equation}

\begin{equation}\label{s3}
    s=\frac{R_f}{R_{in}}\cdot\frac{\%R_{pot}+R}{R_{pot}+R}
    =\frac{2\ k\Omega}{100\ k\Omega}\cdot\frac{440\ \Omega}{490\ \Omega}
    \approx 0.018
\end{equation}

quindi si scelgono:

\begin{itemize}
    \item $R_f = 2\ k\Omega$;
    \item $R_{in} = 100\ k\Omega$;
    \item $R_{pot} = 100\ \Omega$;
    \item $R = 390\ \Omega$;
\end{itemize}

Scegliamo poi $R_{conv}=3.3\ k\Omega$, per avere $3\ mA$ come massimo valore di corrente
$I_{freq}$ (in corrispondenza di $V_{exp}\approx+10\ V$ in uscita dal convertitore
corrente-tensione), se $V_{in}=+8\ V$. Da qui possiamo calcolare il valore di $I_{ref}$:

\begin{equation}\label{iref_calc}
    I_{ref}=I_{freq}e^{\left(-\frac{s\cdot V_{in}}{V_T}\right)}
    =0.003e^{\left(-\frac{0.018\cdot8}{0.026}\right)}
    \approx11.8\ \mu A
\end{equation}

e successivamente, impostando $V_{HR}=+12\ V$ e $V_{LR}=0\ V$ calcoliamo anche $R_{ref}$:

\begin{equation}\label{rref_calc}
    R_{ref}=\frac{V_{HR}}{I_{ref}}=\frac{12}{11.8\cdot10^{-6}}\approx 1\ M\Omega
\end{equation}

La scelta di $R_e$ sarebbe ora condizionata dalla tensione di saturazione dell'operazionale,
poichè altrimenti il circuito smetterebbe di comportarsi come desiderato, cambiando il
riferimento di $V_{LR}$ e quindi anche $I_{ref}$. Avendo collegato la base di $Q_2$ a massa,
il potenziale agli emettitori dovrà rimanere costante a circa $-V_{be}$ per qualsiasi valore
di tensione in ingresso, quindi la tensione in uscita dall'operazionale avrà valore

\begin{equation}
    V_{op}=-V_{be}-R_e(I_{c1}+I_{c2})=%
    -V_{be}-R_e\cdot I_{ref}\left(1+e^{\left(\frac{sV_{in}}{V_T}\right)}\right)\ [V]
\end{equation}

che nel nostro caso dovrà assumere il valore massimo a $V_{in}=+8\ V$, con una corrente di
circa $3\ mA$  in $R_e$. Volendo imporre una tensione di saturazione di valore
$-V_{sat}=-10.5\ V$ calcoliamo quindi il valore di $R_e$.

\begin{equation}
    R_e=\frac{V_{op}+V_{be}}{-I_{ref}\left(1+e^{\left(\frac{sV_{in}}{V_T}\right)}\right)}=
    \frac{-10.5+0.7}{-12\cdot10^{-6}\left(1+e^{\left(\frac{0.018\cdot8}{0.026}\right)}\right)}=
    \frac{9.8}{0.003}\approx 3.3\ k\Omega
\end{equation}

Se però andiamo a calcolare la caduta di potenziale ai capi di $R_e$ corrispondente a $V_{in}=0\ V$,
ovvero:

\begin{equation}\label{vre}
    V_{R_e}=R_e\cdot 2I_{ref}=3.3\cdot10^3\cdot24\cdot10^{-6}\approx79.2\ mV
\end{equation}

notiamo che un valore di resistenza maggiore aumenterebbe la linearità del circuito per bassi
livelli di $V_{in}$, in quanto incrementerebbe la caduta di potenziale sul resistore,
risultando quindi meno influenzata dal rumore. Si modifica allora il circuito cosicché la
resistenza di emettitore sia variabile in modo non-lineare, a seconda della corrente richiesta
in uscita. Questo è reso possibile sostituendo ad $R_e$ un parallelo tra due resistori di
diverso valore, di cui quello più piccolo collegato in serie ad un diodo, l'elemento che si
occuperà effettivamente di modificare il valore di resistenza.

\begin{figure}[H]
    \centering
    \includegraphics{circuits/Rtot_circuit.png}
    \caption{Schema elettrico del resistore di emettitore equivalente}
    \label{Rtot_circuit}
\end{figure}

Si scelgono $R_e'=10\ k\Omega$ e $R_e''=1.2\ k\Omega$.

\begin{equation}\label{rtot}
    R_{tot} =
    \left\{
    \begin{array}{lr}
        R_e'        & \text{con diodo spento, ovvero } I_{ref}+I_{freq}\leq\frac{V_d}{R_e'} \\
        R_e'//R_e'' & \text{con diodo acceso, ovvero } I_{ref}+I_{freq}>\frac{V_d}{R_e'}
    \end{array}
    \right.
\end{equation}

Come conseguenza quindi, avremo che

\begin{equation}\label{vreminmax}
    \left\{
    \begin{array}{ll}
        V_{R_emin}\approx240\ mV & \text{per  } I_{ref}+I_{freq}\approx24\ \mu A\ (V_{in}=0\ V) \\
        V_{R_emax}\approx3\ V    & \text{per } I_{ref}+I_{freq}\approx3\ mA\ (V_{in}=+8\ V)
    \end{array}
    \right.
\end{equation}

Per quanto riguarda i componenti, ancora una volta gli amplificatori operazionali utilizzati
sono dei TL074, mentre per i transistor si sceglie un MPQ3904 \cite{mpq3904}, chip che
ospita 4 unità virtualmente identiche e termicamente accoppiate, e vista la disponibilità, il
diodo viene sostituito con uno dei 4 transistor del chip cortocircuitando base e collettore.

\begin{figure}[H]
    \centering
    \includegraphics{circuits/complete_exponential_converter_circuit.png}
    \caption{Schema elettrico completo del convertitore tensione lineare-esponenziale ($+V_{cc}=+12\ V$)}
    \label{complete_exponential_converter_circuit}
\end{figure}

%--------------------------------------------------------------------------------------------

\subsection*{Risultati Pratici e Misure}

%--------------------------------------------------------------------------------------------

Verifichiamo ora che il comportamento del circuito sia quello desiderato. Possiamo tracciare
la curva di uscita servendoci dell'oscilloscopio in modalità XY e fornendo un segnale a rampa
o triangolare in ingresso, utilizzando quindi il seguente setup:

\begin{figure}[H]
    \centering

    \begin{subfigure}{.5\textwidth}
        \centering
        \includegraphics{block_diagrams/mis_expo_oscilloscope.png}
        \caption{Setup di misura}
        \label{mis_expo_oscilloscope}
    \end{subfigure}%
    \begin{subfigure}{.5\textwidth}
        \centering
        \includegraphics[scale = 0.5]{acquisitions/trans_expo_cut.png}
        \caption{Transcaratteristica}
        \label{expo_transcharacteristic}
    \end{subfigure}

    \caption{Misura della transcaratteristica con oscilloscopio}
\end{figure}

Questa misura ci fornisce una prima conferma che il circuito funzioni come dovrebbe. Il
ginocchio osservabile si forma a ausa della saturazione dell'operazionale utilizzato per il
convertitore corrente-tensione e avviene circa $+8\ V$, valore imposto dai componenti selezionati
precedentemente.

Raccogliamo ora in una tabella i dati per tracciare una transcaratteristica più precisa:

\begin{minipage}{0.49\textwidth}
    \centering
    \includegraphics{block_diagrams/mis_expo.png}
    \captionof{figure}{Setup di misura}
    \label{mis_expo}
\end{minipage}
\begin{minipage}{0.49\textwidth}
    \centering
    \begin{table}[H]
        \centering
        \csvreader[
        tabular = |C||C|L|,
        table head = {\hline \rowcolor{myLightGrey} $V_{in}\ [V]$ & $V_{exp}\ [V]$ & $2^{V_{in}}\ [V]$ \\\hline},
        late after line = \\\hline,
        ]{data/misure_expo.csv}{}{
        \csvcoli & \csvcoliii & \csvcolii
        }
        \caption{Valori misurati}
        \label{expo_table}
    \end{table}
\end{minipage}

\begin{figure}[H]
    \centering

    \begin{subfigure}{.5\textwidth}
        \centering
        \begin{tikzpicture}[scale = 0.85]
            \begin{semilogyaxis}[
                    title = Transcaratteristica in scala logaritmica,
                    no marks,
                    xmin = 0, xmax = 12,
                    ymin = 0.01, ymax = 100,
                    grid = major,
                    grid style = {dashed, gray!30},
                    xlabel = $V_{in}$,
                    ylabel = $V_{exp}$,
                    x unit = \si{\V}, y unit = \si{\V},
                    legend style = {at = {(0.5, -0.25)}, anchor = north},
                    cycle list name = modular,
                ]

                \addplot
                table[x = vin, y = vexp formula, col sep = comma]{./data/misure_expo.csv};

                \addplot
                table[x = vin, y = vexp attesa, col sep = comma]{./data/misure_expo.csv};

                \addplot
                table[x = vin, y = vexp misura, col sep = comma]{./data/misure_expo.csv};

                \legend{Formula \ref{expo_converter}, Valori attesi, Valori misurati}
            \end{semilogyaxis}
        \end{tikzpicture}
    \end{subfigure}%
    \begin{subfigure}{.5\textwidth}
        \centering
        \begin{tikzpicture}[scale = 0.85]
            \begin{axis}[
                    title = Transcaratteristica in scala lineare,
                    no marks,
                    xmin = 0, xmax = 12,
                    ymin = 0, ymax = 15,
                    grid = major,
                    grid style = {dashed, gray!30},
                    xlabel = $V_{in}$,
                    ylabel = $V_{exp}$,
                    x unit = \si{\V}, y unit = \si{\V},
                    legend style = {at = {(0.5, -0.25)}, anchor = north},
                    cycle list name = modular,
                ]

                \addplot
                table[x = vin, y = vexp formula, col sep = comma]{./data/misure_expo.csv};

                \addplot
                table[x = vin, y = vexp attesa, col sep = comma]{./data/misure_expo.csv};

                \addplot
                table[x = vin, y = vexp misura, col sep = comma]{./data/misure_expo.csv};

                \legend{Formula \ref{expo_converter}, Valori attesi, Valori misurati}
            \end{axis}
        \end{tikzpicture}
    \end{subfigure}

    \caption{Grafici delle misure riportate in tabella \ref{expo_table}}
    \label{expo_graphs}
\end{figure}

In azzurro possiamo vedere i valori calcolati tramite la formula \ref{expo_converter},
in rosso invece i valori ottenuti moltiplicando il dato misurato a $0\ V$ per $2^{V_{in}}$
e infine in nero i valori effettivamente misurati. Si vede che rispetto a quanto calcolato
c'è un lieve discostamento dovuto con ogni probabilità alla tolleranza dei valori dei
componenti utilizzati, tuttavia le misure risultano quasi coincidenti con la curva esponenziale
di pendenza $2$ (in rosso), quindi possiamo affermare che il circuito funziona come desiderato.

Ancora una volta, il ginocchio presente subito dopo i $10\ V$ è dovuto alla saturazione
dell'operazionale nel convertitore corrente-tensione, che con una alimentazione di $\pm12\ V$
presenta dei valori di saturazione di circa $\pm10.5\ V$.

Si verifica anche che la curva della transcaratteristica in scala lineare coincide con la
transcaratteristica tracciata con l'oscilloscopio (figura \ref{expo_transcharacteristic}).

%--------------------------------------------------------------------------------------------

\section{Somma di più Ingressi}

%--------------------------------------------------------------------------------------------

Per fare in modo che possa essere utilizzato un segnale in tensione come modulante (ingresso
LFO), si aggiunge un resistore uguale a $R_{in}$ all'ingresso invertente dell'amplificatore,
modificando quindi la relazione \ref{amplifier} nel seguente modo:

\begin{equation}
    V_{b1}=-s\cdot\sum_0^n{V_{n}}\ [V]
\end{equation}

\begin{figure}[H]
    \centering
    \includegraphics{circuits/summer_circuit.png}
    \caption{Circuito sommatore invertente}
    \label{summer_circuit}
\end{figure}

e rendendo l'amplificatore un sommatore invertente. Si noti che idealmente non c'è un limite
al numero di ingressi che è possibile aggiungere.

Per la modulazione manuale si utilizzano due potenziometri, uno per una regolazione grossolana
e uno per una regolazione più fine, e li si collega ad uno degli ingressi del sommatore nel
seguente modo:

\begin{figure}[H]
    \centering
    \includegraphics{circuits/coarse_fine_circuit.png}
    \caption{Circuito per la regolazione manuale della frequenza ($\pm V_{cc}=\pm12\ V$)}
    \label{coarse_fine_circuit}
\end{figure}

questo circuito permette di sommare (o sottrarre) una tensione compresa nel range di
alimentazione, inoltre la manopola "Fine" consente di regolare la tensione in modo più
preciso attorno al punto selezionato con "Coarse".

Infine, il segnale modulante d'ingresso viene portato ad un altro degli ingressi del sommatore,
attraverso un potenziometro per la regolazione del volume.

\begin{figure}[H]
    \centering
    \includegraphics{circuits/lfo_input_circuit.png}
    \caption{Circuito di ingresso del segnale modulante}
    \label{lfo_input_circuit}
\end{figure}

%--------------------------------------------------------------------------------------------

\section{Raddrizzatore}

%--------------------------------------------------------------------------------------------

Come visto dalla formula \ref{expo_converter}, la tensione in ingresso $V_{in}$ deve essere di
valore positivo per polarizzare correttamente il transistor, e volendo inserire nel circuito
un nodo sommatore per avere la possibilità di utilizzare un segnale in tensione come modulante,
dobbiamo assicurarci che $V_{b1}$ non diventi mai positiva (il segno viene invertito
dall'amplificatore). Si aggiunge quindi un blocco raddrizzatore, realizzato con un diodo e
un operazionale, che compenserà per la caduta di tensione sul diodo. Lo schema utilizzato
è il seguente:

\begin{figure}[H]
    \centering
    \includegraphics{circuits/clipper_circuit.png}
    \caption{Circuito raddrizzatore}
    \label{clipper_circuit}
\end{figure}

Il resistore collegato a $V_{b1}'$ è necessario per il ricircolo della corrente di
polarizzazione del diodo, che altrimenti non si accenderebbe mai. Viene inoltre aggiunto un
LED per visualizzare quando il raddrizzatore è in funzione, in modo da permettere all'utente
di correggere eventuali errori nel segnale modulante in ingresso.

Quindi in uscita si avrà la stessa tensione in ingresso se negativa, mentre un valore molto
prossimo a $0$ se positiva.

\begin{equation}\label{clipper_out}
    V_{out} =
    \left\{
    \begin{array}{lr}
        V_{in} & \text{con } V_{in}\le0 \\
        0      & \text{con } V_{in}>0
    \end{array}
    \right.
\end{equation}

\begin{equation}\label{clipper_op}
    V_{op} =
    \left\{
    \begin{array}{lr}
        V_{in}+V_d & \text{con } V_{in}\le0\Rightarrow \text{LED spento} \\
        +V_{sat}   & \text{con } V_{in}>0\Rightarrow \text{LED acceso}
    \end{array}
    \right.
\end{equation}

Il diodo utilizzato per lo scopo è un 1N4148 \cite{1n4148}, un comunissimo diodo per piccoli
segnali, mentre l'operazionale è sempre un TL074.

%--------------------------------------------------------------------------------------------

\subsection*{Risultati Pratici e Misure}

%--------------------------------------------------------------------------------------------

Anche in questo caso possiamo tracciare la transcaratteristica con l'oscilloscopio, sfruttando
la modalità XY, il setup è molto simile a quello per il convertitore lineare-esponenziale:

\begin{figure}[H]
    \centering

    \begin{subfigure}{.5\textwidth}
        \centering
        \includegraphics{block_diagrams/mis_clipper_oscilloscope.png}
        \caption{Setup di misura}
        \label{mis_clipper_oscilloscope}
    \end{subfigure}%
    \begin{subfigure}{.5\textwidth}
        \centering
        \includegraphics[scale = 0.5]{acquisitions/trans_clipper_cut.png}
        \caption{Transcaratteristica}
        \label{clipper_transcharacteristic}
    \end{subfigure}

    \caption{Misura della transcaratteristica con oscilloscopio}
\end{figure}

Dopo aver apprezzato il corretto funzionamento si passa alla raccolta dei dati per un grafico
più preciso.

\begin{figure}[H]
    \centering
    \includegraphics{block_diagrams/mis_clipper.png}
    \caption{Setup di misura}
    \label{mis_clipper}
\end{figure}

\begin{minipage}{0.45\textwidth}
    \centering
    \begin{table}[H]
        \centering
        \csvreader[
        tabular = |C||C|L|,
        table head = {\hline \rowcolor{myLightGrey} $V_{in}\ [V]$ & $V_{out}\ [V]$ & $V_{op}\ [V]$ \\\hline},
        late after line = \\\hline,
        ]{data/misure_clipper.csv}{}{
        \csvcoli & \csvcoliii & \csvcolv
        }
        \caption{Tabella dei dati raccolti}
        \label{clipper_table}
    \end{table}
\end{minipage}
\begin{minipage}{0.45\textwidth}
    \centering
    \begin{figure}[H]
        \centering
        \begin{tikzpicture}[scale = 0.85]
            \begin{axis}[
                    title = Transcaratteristica,
                    no marks,
                    xmin = -7, xmax = 7,
                    ymin = -7, ymax = 7,
                    grid = major,
                    grid style = {dashed, gray!30},
                    xlabel = $V_{in}$,
                    ylabel = $V_{out}$,
                    x unit = \si{\V}, y unit = \si{\V},
                    legend style = {at = {(0.5, -0.25)}, anchor = north},
                    cycle list name = modular,
                ]

                \addplot
                table[x = vin, y = vout attesa, col sep = comma]{./data/misure_clipper.csv};

                \addplot
                table[x = vin, y = vout misura, col sep = comma]{./data/misure_clipper.csv};

                \legend{Formula \ref{clipper_out}, Valori misurati}
            \end{axis}
        \end{tikzpicture}
    \end{figure}

    \begin{figure}[H]
        \begin{tikzpicture}[scale = 0.85]
            \begin{axis}[
                    title = Uscita all'operazionale,
                    no marks,
                    xmin = -7, xmax = 7,
                    ymin = -7, ymax = 12,
                    grid = major,
                    grid style = {dashed, gray!30},
                    xlabel = $V_{in}$,
                    ylabel = $V_{op}$,
                    x unit = \si{\V}, y unit = \si{\V},
                    legend style = {at = {(0.5, -0.25)}, anchor = north},
                    cycle list name = modular,
                ]

                \addplot
                table[x = vin, y = vop attesa, col sep = comma]{./data/misure_clipper.csv};

                \addplot
                table[x = vin, y = vop misura, col sep = comma]{./data/misure_clipper.csv};

                \legend{Formula \ref{clipper_op}, Valori misurati}
            \end{axis}
        \end{tikzpicture}
    \end{figure}
\end{minipage}

Possiamo quindi affermare che il circuito appena discusso si comporta esattamente come
desiderato, a meno della tensione di saturazione dell'operazionale, cosa certamente
trascurabile in quanto permette comunque l'accensione del LED.

%--------------------------------------------------------------------------------------------

\chapter{Modalità di Funzionamento}

%--------------------------------------------------------------------------------------------

Per quanto riguarda la selezione della modalità di funzionamento, è facile intuire che basta
dividere $f_{clk}$ ulteriormente per avere un range di frequenze diverso, come già visto nel
capitolo \ref{segnali_principali}. Quindi aggiungiamo un blocco divisore di frequenza prima
che il segnale arrivi ai contatori e facciamo in modo che la scelta sia determinata da un
segnale logico.

\begin{figure}[H]
    \centering
    \includegraphics{block_diagrams/mode_selector_block_diagram.png}
    \caption{Schema a blocchi del sottosistema per la selezione della modalità}
    \label{mode_selector_block_diagram}
\end{figure}

Lo schema a blocchi sopra riportato si traduce dunque nel seguente circuito, in cui $f_{clk}$
viene divisa per $256$, traslando il range di funzionamento nell'intervallo
$(\approx0.11\ Hz\div27.5\ Hz)$.

\begin{figure}[H]
    \centering
    \includegraphics{circuits/mode_selector_circuit.png}
    \caption{Circuito per la selezione della modalità ($V_{dd}=V_{HR}=+5\ V$)}
    \label{mode_selector_circuit}
\end{figure}

L'unico componente nuovo è un circuito integrato che ospita 4 porte logiche AND, il 74LS08
\cite{74ls08}, utilizzate per il multiplexing del segnale di clock.

Si aggiungono anche dei LED per la visualizzazione dello stato e delle resistenze di pull-down
all'ingresso abilitante delle porte logiche (non rappresentati nello schema in figura
\ref{mode_selector_circuit}).

%--------------------------------------------------------------------------------------------

\chapter{Generazione dei Segnali Secondari}\label{segnali_secondari}

TODO


\section{Onda Quadra}

TODO


\section{Dente di Sega}

TODO


\section{Sinusoide}

TODO


\section{Impulso}

TODO
\chapter{Stadi di Uscita}

%--------------------------------------------------------------------------------------------

Per quanto rigurarda gli stadi di uscita, si collega in serie un filtro passa-basso attivo,
che avrà anche lo scopo di isolare l'assorbimento di corrente dai circuiti che generano i
segnali.

\begin{figure}[H]
    \centering
    \includegraphics{circuits/active_filter_circuit.png}
    \caption{Circuito del filtro attivo utilizzato}
    \label{active_filter_circuit}
\end{figure}

sfruttando questa soluzione circuitale infatti, tutta la corrente prelevata dal circuito in
uscita verrà fornita dagli amplificatori operazionali. Le relazioni del circuito sono le
seguenti:

\begin{equation}\label{active_filter}
    V_{out}=-V_{in}\frac{R_f}{R_{in}}\cdot\frac{1}{1+j\omega R_fC}\ [V]
\end{equation}

\begin{equation}\label{fcut}
    f_{cut}=\frac{1}{2\pi R_fC}\ [Hz]
\end{equation}

\begin{equation}\label{gain1}
    A_{dB}=20log_{10}\left(\left|\frac{R_f}{R_{in}}\cdot\frac{1}{1+j\omega R_fC}\right|\right)\ [dB]
\end{equation}

oppure, dai valori misurati:

\begin{equation}\label{gain2}
    A_{dB}=20log_{10}\left(\frac{V_{rms\_out}}{V_{rms\_in}}\right)\ [dB]
\end{equation}

La frequenza di taglio viene presa attorno ai $50\ kHz$ per conservare tutto lo spettro audio
e rimuovere invece disturbi in alta frequenza dovuti ad esempio ai segnali di clock.
Scegliendo $R_f=100\ k\Omega$ quindi, il valore del condensatore va preso di circa $30\ pF$,
mentre $R_{in}$ deve avere valore pari a $R_f$. In questo modo si ottiene guadagno unitario e

\begin{equation}
    f_{cut}=\frac{1}{2\pi\cdot100\cdot10^3\cdot30\cdot10^-12}\approx53\ kHz
\end{equation}

Il setup di misura viene riportato in figura \ref{mis_filter}. Per la verifica del funzionamento
si misura il $V_{rms}$ di un segnale sinusoidale in ingresso e in uscita al filtro, agendo
sulla frequenza del suddetto segnale.

\begin{figure}[H]
    \centering
    \includegraphics{block_diagrams/mis_filter.png}
    \caption{Setup di misura}
    \label{mis_filter}
\end{figure}

Successivamente, con la formula \ref{gain2} si calcola il valore del guadagno. Infine i dati
vengono raccolti in tabella, riportati in grafico e confrontati con i valori teorici calcolati
con la formula \ref{gain1}.

\begin{table}[H]
    \centering
    \csvreader[
    tabular = |C|C||C|L|,
    table head = {\hline \rowcolor{myLightGrey} $f_{in}\ [kHz]$ & $V_{rmsin}\ [V]$ & $V_{rmsout}\ [V]$ & $A_{dB}\ [dB]$\\\hline},
    late after line = \\\hline,
    ]{data/misure_bode.csv}{}{
    \csvcoli & \csvcolii & \csvcoliii & \csvcolv
    }
    \caption{Misure del guadagno del filtro attivo}
    \label{filter_table}
\end{table}

\begin{figure}[H]
    \centering
    \begin{tikzpicture}
        \centering
        \begin{semilogxaxis}[
                title = Diagramma di bode del filtro,
                no marks,
                width = 0.95\textwidth,
                height = 0.45\textwidth,
                xmin = 0.1, xmax = 4000,
                ymin = -40, ymax = 5,
                grid = major,
                grid style = {dashed, gray!30},
                xlabel = $f_{in}$,
                ylabel = $A_{dB}$,
                x unit = \si{\kHz}, y unit = \si{\dB},
                legend style = {at = {(0.5, -0.25)}, anchor = north},
                cycle list name = modular,
            ]

            \addplot
            table[x = fin, y = A dB calcolato, col sep = comma]{./data/misure_bode.csv};

            \addplot
            table[x = fin, y = A dB misurato, col sep = comma]{./data/misure_bode.csv};

            \legend{Formula \ref{gain1}, Formula \ref{gain2} (con i valori misurati)}
        \end{semilogxaxis}
    \end{tikzpicture}
\end{figure}

\begin{figure}[H]
    \centering

    \begin{subfigure}{.5\textwidth}
        \centering
        \includegraphics[scale = 0.2]{acquisitions/filter_10kHz.png}
        \caption{$f_{in}=10\ kHz$}
        \label{acq_filter_10kHz}
    \end{subfigure}%
    \begin{subfigure}{.5\textwidth}
        \centering
        \includegraphics[scale = 0.2]{acquisitions/filter_100kHz.png}
        \caption{$f_{in}=100\ kHz$}
        \label{acq_filter_100kHz}
    \end{subfigure}
    \begin{subfigure}{.5\textwidth}
        \centering
        \includegraphics[scale = 0.2]{acquisitions/filter_500kHz.png}
        \caption{$f_{in}=500\ kHz$}
        \label{acq_filter_500kHz}
    \end{subfigure}%
    \begin{subfigure}{.5\textwidth}
        \centering
        \includegraphics[scale = 0.2]{acquisitions/filter_1MHz.png}
        \caption{$f_{in}=1\ MHz$}
        \label{acq_filter_MHz}
    \end{subfigure}

    \caption{Acquisizioni dei segnali in ingresso e uscita al filtro per diversi valori di $f_{in}$}
    \label{acq_filter}
\end{figure}

In serie ai filtri per le onde a rampa e dente di sega viene anche collegato un amplificatore
invertente con guadagno unitario, con schema uguale a quello rappresentato in figura \ref{inverting_amp_circuit}
per riportare le onde alla loro forma originale, in quanto non sono simmetriche, come lo sono
invece le altre 3.

Infine, in serie a tutte le uscite degli ultimi operazionali prima del connettore, si collegano
delle resistenze di protezione, dal valore di circa $1\ k\Omega$ per limitare la corrente in
uscita in caso di eventuali cortocircuiti, che potrebbero provocare danni agli amplificatori
operazionali, e in ingresso i segnali vengono collegati ad un potenziometro per la regolazione
del volume in uscita.

\begin{figure}[H]
    \centering

    \begin{subfigure}{\textwidth}
        \centering
        \includegraphics{circuits/inverting_output_stage_circuit.png}
        \caption{Circuito utilizzato per triangolo, sinusoide e onda quadra}
        \label{inverting_output_stage_circuit}
    \end{subfigure}
    \begin{subfigure}{\textwidth}
        \centering
        \includegraphics{circuits/noninverting_output_stage_circuit.png}
        \caption{Circuito utilizzato per rampa e dente di sega}
        \label{noninverting_output_stage_circuit}
    \end{subfigure}

    \caption{Stadi d'uscita completi}
    \label{output_stages}
\end{figure}

%--------------------------------------------------------------------------------------------

%\chapter{Protezione del Circuito}

TODO
%\chapter{Composizione delle Schede}

TODO
\unchapter{Considerazioni Finali e Conclusioni}

\lipsum[2-4]

% aggiungere appendice con schemi elettrici

% bibliografia
\printbibliography
\addcontentsline{toc}{chapter}{Bibliografia}

% fine del documento
\end{document}