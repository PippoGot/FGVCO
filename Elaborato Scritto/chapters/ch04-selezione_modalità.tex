\chapter{Modalità di Funzionamento}

%--------------------------------------------------------------------------------------------

Per quanto riguarda la selezione della modalità di funzionamento, è facile intuire che basta
dividere $f_{clk}$ ulteriormente per avere un range di frequenze diverso, come già visto nel
capitolo \ref{segnali_principali}. Quindi aggiungiamo un blocco divisore di frequenza prima
che il segnale arrivi ai contatori e facciamo in modo che la scelta sia determinata da un
segnale logico.

\begin{figure}[H]
    \centering
    \includegraphics{block_diagrams/mode_selector_block_diagram.png}
    \caption{Schema a blocchi del sottosistema per la selezione della modalità}
    \label{mode_selector_block_diagram}
\end{figure}

Lo schema a blocchi sopra riportato si traduce dunque nel seguente circuito, in cui $f_{clk}$
viene divisa per $256$, traslando il range di funzionamento nell'intervallo
$(\approx0.11\ Hz\div27.5\ Hz)$.

\begin{figure}[H]
    \centering
    \includegraphics{circuits/mode_selector_circuit.png}
    \caption{Circuito per la selezione della modalità ($V_{dd}=V_{HR}=+5\ V$)}
    \label{mode_selector_circuit}
\end{figure}

L'unico componente nuovo è un circuito integrato che ospita 4 porte logiche AND, il 74LS08
\cite{74ls08}, utilizzate per il multiplexing del segnale di clock.

Si aggiungono anche dei LED per la visualizzazione dello stato e delle resistenze di pull-down
all'ingresso abilitante delle porte logiche (non rappresentati nello schema in figura
\ref{mode_selector_circuit}).

%--------------------------------------------------------------------------------------------
