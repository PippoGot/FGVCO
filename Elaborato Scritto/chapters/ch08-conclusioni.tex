\unchapter{Conclusioni}

%--------------------------------------------------------------------------------------------

\unsection{Suddivisione delle Schede}

%--------------------------------------------------------------------------------------------

Ogni blocco discusso finora viene collegato assieme, ottenendo lo schema seguente:

\begin{figure}[H]
    \centering
    \includegraphics[scale = 0.5]{block_diagrams/full_block_diagram.png}
    \caption{Schema a blocchi completo}
    \label{full_block_diagram}
\end{figure}

Si suddivide poi l'intero circuito in un totale di 3 schede, in modo da ridurre al minimo
l'ingombro orizzontale del modulo e allo stesso tempo non avere delle schede troppo dense
di componenti.

\begin{figure}[H]
    \centering
    \includegraphics[scale = 0.5]{block_diagrams/full_block_diagram_colored.png}
    \caption{Suddivisione dei blocchi nelle varie schede}
    \label{full_block_diagram_colored}
\end{figure}

Le schede verranno impilate una sopra l'altra e collegate tra loro con degli appositi connettori.
L'interfaccia mostrata in figura \ref{panel_explained} sarà effettivamente il pannello
frontale del nostro modulo, subito sotto andrà la scheda di interfaccia, la quale ospiterà
tutti i componenti atti all'interazione con l'utente e gli altri moduli (manopole, connettori
jack, LED, interruttori, ecc...). Nelle altre due schede invece si collocherà tutto il resto
del circuito secondo quanto specificato in figura \ref{full_block_diagram_colored}.

\begin{figure}[H]
    \centering
    \includegraphics[scale = 0.9]{misc/boards_layout.png}
    \caption{Composizione schede}
    \label{boards_layout}
\end{figure}

Gli schemi elettrici finali utilizzati sono riportati nell'appendice \ref{circuit_diagrams}.

%--------------------------------------------------------------------------------------------

\unsection{Possibili Impieghi del Modulo}

%--------------------------------------------------------------------------------------------

Il modulo in sè non risulta comodamente utilizzabile da solo e va integrato in un sistema
adeguato assieme a filtri, amplificatori e altri diversi moduli con funzioni varie ed eventuali,
oltre ovviamente ad un sistema per la riproduzione dell'audio. Si riportano allora degli
esempi di utilizzo:

\begin{figure}[H]
    \centering
    \includegraphics{block_diagrams/example_setup_1.png}
    \caption{Setup d'esempio 1}
    \label{example_setup_1}
\end{figure}

In questo setup si fa uso di un sequenziatore, in grado di fornire un segnale di controllo
che andrà a determinare la nota generata dall'oscillatore. Successivamente il segnale grezzo
generato dall'oscillatore viene modificato da un filtro controllabile in tensione (Voltage
Controlled Filter, o VCF) e poi fornito ad un sistema per la riproduzione del suono.

\begin{figure}[H]
    \centering
    \includegraphics{block_diagrams/example_setup_2.png}
    \caption{Setup d'esempio 2}
    \label{example_setup_2}
\end{figure}

Qui invece, la nota generata rimane costante nel tempo, ma attraverso una combinazione di un
LFO e un amplificatore controllato in tensione (Voltage Controlled Amplifier, o VCA), si
modifica il volume del suono che viene poi filtrato e riprodotto.

\begin{figure}[H]
    \centering
    \includegraphics{block_diagrams/example_setup_3.png}
    \caption{Setup d'esempio 3}
    \label{example_setup_3}
\end{figure}

In questo ultimo esempio, il sequenziatore determina la nota generata dal VCO e la frequenza
di taglio del filtro. Inoltre l'uscita "Gate" viene data in pasto ad un generatore di inviluppo,
che si occupa di creare un particolare segnale composto da 4 fasi, ovvero attacco, discesa,
sostegno e rilascio, che viene poi utilizzato per il controllo del volume della nota.
In questo modo è quindi possibile simulare la pressione di un tasto e dare un aspetto più
naturale al suono. Ancora una volta infine, il segnale viene filtrato e poi riprodotto tramite
un adeguato sistema.

%--------------------------------------------------------------------------------------------

\unsection{Considerazioni Finali}

%--------------------------------------------------------------------------------------------

Come già accennato nell'introduzione, la progettazione del modulo sarebbe stata largamente
semplificata se si fossero utilizzati dei microcontrollori, tuttavia l'implementazione
analogica realizzata risulta comunque ottima per lo scopo. È ad ogni modo possibile progettare
una scheda con microcontrollori che realizza le stesse funzioni di quelle discusse in questa
tesi, compatibile con la scheda "interfaccia" e che può quindi essere eventualmente sostituita
al corrispettivo analogico, sfruttando la divisione delle schede scelta.

\begin{figure}[H]
    \centering
    \includegraphics[scale = 0.9]{misc/boards_layout_microcontroller.png}
    \caption{Composizione schede con scheda a microcontrollore}
    \label{boards_layout_microcontroller}
\end{figure}

Per quanto riguarda l'obiettivo, possiamo dire che le specifiche di progetto fornite
nell'introduzione sono state pienamente rispettate a meno della precisione sulle
note prodotte che risultano in un range leggermente diverso da quello desiderato.

\begin{table}[H]
    \centering
    \csvreader[
    tabular = |I||I|I|I|U|,
    table head = {\hline \rowcolor{myLightGrey} $V_{in}\ [V]$ & $VCO\ [Hz]$ & $LFO\ [Hz]$ & $VCO_{aim}\ [Hz]$ & $LFO_{aim}\ [Hz]$\\\hline},
    late after line = \\\hline,
    ]{data/misure_frequenza.csv}{}{
    \csvcoli & \csvcolii & \csvcoliii & \csvcolvi & \csvcolvii
    }
    \caption{Valori di frequenze ottenuti e desiderati}
    \label{freq_table}
\end{table}

\begin{figure}[H]
    \centering
    \begin{tikzpicture}
        \centering
        \begin{semilogyaxis}[
                no marks,
                width = 0.95\textwidth,
                height = 0.45\textwidth,
                xmin = 0, xmax = 11,
                ymin = 0.1, ymax = 100000,
                grid = major,
                grid style = {dashed, gray!30},
                xlabel = $V_{in}$,
                ylabel = $f_{out}$,
                x unit = \si{\V}, y unit = \si{\Hz},
                legend style = {at = {(0.5, -0.25)}, anchor = north},
                cycle list name = modular,
            ]

            \addplot
            table[x = vin, y = fout vco, col sep = comma]{./data/misure_frequenza.csv};

            \addplot
            table[x = vin, y = fout lfo, col sep = comma]{./data/misure_frequenza.csv};

            \addplot
            table[x = vin, y = vco voluto, col sep = comma]{./data/misure_frequenza.csv};

            \addplot
            table[x = vin, y = lfo voluto, col sep = comma]{./data/misure_frequenza.csv};

            \legend{Colonna 1, Colonna 3, Colonna 2, Colonna 4}
        \end{semilogyaxis}
    \end{tikzpicture}
    \caption{Frequenza di uscita del modulo in modalità VCO e LFO}
    \label{module_freq_graph}
\end{figure}

Tuttavia un aspetto che meriterebbe più attenzioni è l'effetto della temperatura dell'ambiente
sul convertitore lineare-esponenziale, in quanto effettivamente $V_T$ è funzione di essa.
Tale problema potrebbe essere risolto con l'impiego di una termoresistenza adeguatamente
dimensionata, ma non si scenderà oltre in dettaglio.

%--------------------------------------------------------------------------------------------

\unsection{Risultato}

%--------------------------------------------------------------------------------------------

Per concludere si riportano delle foto del risultato finale ottenuto.

\begin{figure}[H]
    \centering

    \begin{subfigure}{.4\textwidth}
        \centering
        \includegraphics[height = \textwidth]{misc/module_result.jpeg}
        \caption{Modulo realizzato}
        \label{module_result}
    \end{subfigure}%
    \begin{subfigure}{.4\textwidth}
        \centering
        \includegraphics[height = \textwidth]{misc/module_result_side.jpeg}
        \caption{Vista lato}
        \label{module_result_side}
    \end{subfigure}

    \caption*{}
\end{figure}

%--------------------------------------------------------------------------------------------
