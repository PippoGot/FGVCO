\chapter{Convertitore Tensione-Frequenza}

%--------------------------------------------------------------------------------------------

Si vuole ora progettare il circuito per la generazione del segnale $f_{clk}$, con le
specifiche ottenute dai capitoli precedenti, ovvero:

\begin{itemize}
    \item Frequenza minima: $\approx 14\ kHz$;
    \item Frequenza massima: $\approx 3.6\ MHz$;
    \item Livello logico basso: $0\ V$;
    \item Livello logico alto: $+5\ V$;
\end{itemize}

%--------------------------------------------------------------------------------------------

\subsection*{Principio di Funzionamento}

%--------------------------------------------------------------------------------------------

Ciò di cui abbiamo bisogno è un circuito in grado di convertire una tensione in un segnale
a onda rettangolare con frequenza proporzionale alla tensione stessa, ovvero un convertitore
tensione-frequenza.

In commercio è possibile trovare diversi chip in grado di svolgere questa funzione
semplicemente aggiungendo una manciata di componenti di contorno, anche se la maggior
parte di questi non arriva a coprire l'intero range di funzionamento di cui abbiamo bisogno
(come ad esempio il noto LM331 \cite{lm331}). Nel nostro caso si utilizza un VFC110
\cite{vfc110}, circuito integrato specializzato che vanta un'ottima linearità e la capacità
di fornire una frequenza massima di $4\ MHz$ in uscita con una corrispondente tensione in
ingresso di $+10\ V$, esattamente ciò che la nostra applicazione richiede.

\begin{figure}[H]
    \centering
    \includegraphics{circuits/vfc110_internal.png}
    \caption{Estratto utile della struttura interna di un VFC110}
    \label{vfc110_internal}
\end{figure}

Il cuore del circuito (figura \ref{vfc110_internal}) consiste in un operazionale configurato
come integratore, con tensione di uscita proporzionale alla carica immagazzinata nella sua
capacità di feedback $C_{int}$. Una tensione in ingresso $V_{in}$ sviluppa una corrente
$I_{in}=\frac{V_{in}}{R_{in}}$ che viene forzata in $C_{int}$, causando quindi una rampa
decrescente in uscita. Arrivati a $0\ V$ il comparatore scatta, attivando il timer one-shot.
Quindi un generatore di corrente $I_{ref}$ (dal valore di circa $1\ mA$) viene connesso
all'ingresso dell'integratore per un periodo di durata pari a $T_{OS}$, causando una rampa
crescente in uscita. Infine il ciclo ricomincia.

\begin{figure}[H]
    \centering
    \includegraphics{graphs/integrator_behaviour.png}
    \caption{Forme d'onda teoriche del VFC110}
    \label{integrator_behaviour}
\end{figure}

Per uno studio più approfondito sul funzionamento del VFC110 si consiglia la lettura del
datasheet del componente, dal quale si ricava la configurazione del circuito utilizzato
per sfruttare l'intero range offerto (figura \ref{VFC_circuit}). Si modificano solo i valori
di alimentazione, sostituendoli con quelli dello standard scelto.

\begin{figure}[H]
    \centering
    \includegraphics{circuits/VFC_circuit.png}
    \caption{Schema elettrico del circuito utilizzato ($\pm V_{cc}=\pm 12\ V$, $+V_{}=+5\ V$)}
    \label{VFC_circuit}
\end{figure}

Si noti che gli unici componenti aggiunti sono condensatori di filtro e un resistore di
pull-up per l'uscita a collettore aperto.

Si riportano anche le relazioni tra le principali grandezze in gioco:

\begin{equation}\label{duty_cycle}
    I_{in}=I_{ref}\cdot\delta\ [A]
    \qquad
    \rightarrow
    \qquad
    \delta=\frac{I_{in}}{I_{ref}}=\frac{V_{in}}{R_{in}\cdot I_{ref}}
\end{equation}

\begin{equation}\label{fout}
    \frac{V_{in}}{R_{in}}=I_{ref}\cdot f_{out}\cdot T_{OS}\ [A]
    \qquad
    \rightarrow
    \qquad
    f_{out}=\frac{V_{in}}{R_{in}\cdot I_{ref}\cdot T_{OS}}=\frac{\delta}{T_{OS}}\ [Hz]
\end{equation}

%--------------------------------------------------------------------------------------------

\subsection*{Risultati Pratici e Misure}

%--------------------------------------------------------------------------------------------

Si procede quindi alla verifica del corretto funzionamento del circuito. Il setup di
misura utilizzato è il seguente:

\begin{figure}[H]
    \centering
    \includegraphics{block_diagrams/mis_VFC.png}
    \caption{Setup di misura del VFC}
    \label{mis_VFC}
\end{figure}

Successivamente si raccolgono in tabella i dati che verranno poi utilizzati per tracciare i
grafici corrispondenti:

\begin{table}[H]
    \centering
    \csvreader[
    tabular = |C||C|L|,
    table head = {\hline \rowcolor{myLightGrey} $V_{in}\ [V]$ & $\delta\ [\%]$ & $f_{out}\ [kHz]$ \\\hline},
    late after line = \\\hline,
    ]{data/misure_vfc.csv}{}{
    \csvcoli & \csvcoliii & \csvcolv
    }
    \caption{Valori misurati del blocco VFC}
    \label{vfc_table}
\end{table}

Possiamo anzitutto verificare che il comportamento dell'integratore corrisponde a quello
descritto nel paragrafo precedente, e si nota che la durata del periodo basso di $f_{out}$
si estende lungo il tratto di rampa crescente come in figura \ref{integrator_behaviour},
ovvero durante il periodo $T_{OS}$, in cui il timer risulta attivo. Possiamo anche
notare che la sua durata non varia in funzione di $V_{in}$ o $f_{out}$, ma rimane invece
pressoché costante a circa $100\ ns$. Questo valore risulta corretto se inserito nella
formula \ref{fout} poichè fornisce una stima accurata dei valori di frequenza misurati.

\begin{figure}[H]
    \centering

    \begin{subfigure}{.5\textwidth}
        \centering
        \includegraphics[scale = 0.2]{acquisitions/VFC_1V.png}
        \caption{$V_{in}=1\ V$}
        \label{acq_vfc110_1v}
    \end{subfigure}%
    \begin{subfigure}{.5\textwidth}
        \centering
        \includegraphics[scale = 0.2]{acquisitions/VFC_5V.png}
        \caption{$V_{in}=5\ V$}
        \label{acq_vfc110_5v}
    \end{subfigure}
    \begin{subfigure}{.5\textwidth}
        \centering
        \includegraphics[scale = 0.2]{acquisitions/VFC_10V.png}
        \caption{$V_{in}=10\ V$}
        \label{acq_vfc110_10v}
    \end{subfigure}

    \caption{Acquisizioni dei segnali di interesse per diversi valori di $V_{in}$}
    \label{acq_vfc110}
\end{figure}

\begin{figure}[H]
    \centering

    \begin{subfigure}{.5\textwidth}
        \centering
        \begin{tikzpicture}[scale = 0.85]
            \begin{axis}[
                    title = Duty Cycle,             % title
                    no marks,
                    xmin = 0, xmax = 12,            % limit values
                    ymin = 0, ymax = 50,
                    grid = major,                   % grid
                    grid style = {dashed, gray!30},
                    xlabel = $V_{in}$,              % axis titles and units
                    ylabel = $\delta$,
                    x unit = \si{\V}, y unit = \si{\percent},
                    legend style = {at = {(0.5, -0.25)}, anchor = north},
                    cycle list name = modular,
                ]

                \addplot
                table[x = vin, y = duty atteso, col sep = comma]{./data/misure_vfc.csv};

                \addplot
                table[x = vin, y = duty misurato, col sep = comma]{./data/misure_vfc.csv};

                \legend{Formula \ref{duty_cycle}, Valori misurati}
            \end{axis}
        \end{tikzpicture}
    \end{subfigure}%
    \begin{subfigure}{.5\textwidth}
        \centering
        \begin{tikzpicture}[scale = 0.85]
            \begin{axis}[
                    title = Frequenza,              % title
                    no marks,
                    xmin = 0, xmax = 12,            % limit values
                    ymin = 0, ymax = 5000,
                    grid = major,                   % grid
                    grid style = {dashed, gray!30},
                    xlabel = $V_{in}$,              % axis titles and units
                    ylabel = $f_{out}$,
                    x unit = \si{\V}, y unit = \si{\kHz},
                    legend style = {at = {(0.5, -0.25)}, anchor = north},
                    cycle list name = modular,
                ]

                \addplot
                table[x = vin, y = fout attesa, col sep = comma]{./data/misure_vfc.csv};

                \addplot
                table[x = vin, y = fout misurata, col sep = comma]{./data/misure_vfc.csv};

                \legend{Formula \ref{fout}, Valori misurati}
            \end{axis}
        \end{tikzpicture}
    \end{subfigure}

    \caption{Grafici delle grandezze riportate in tabella \ref{vfc_table}}
    \label{vfc_graphs}
\end{figure}

Si nota che l'andamento del duty cycle si distacca leggermente dalla teoria, pur avendo
comunque un comportamento lineare. Questo è dovuto probabilmente a causa di diverse
non-idealità del componente, come ad esempio le tolleranze dei componenti interni al
VFC110, tuttavia il parametro di maggiore interesse in questo caso è $f_{out}$, che invece
risulta quasi uguale ai valori calcolati, come evidente in figura \ref{vfc_graphs}b.

%--------------------------------------------------------------------------------------------
